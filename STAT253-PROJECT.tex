% Options for packages loaded elsewhere
\PassOptionsToPackage{unicode}{hyperref}
\PassOptionsToPackage{hyphens}{url}
%
\documentclass[
]{article}
\usepackage{amsmath,amssymb}
\usepackage{lmodern}
\usepackage{ifxetex,ifluatex}
\ifnum 0\ifxetex 1\fi\ifluatex 1\fi=0 % if pdftex
  \usepackage[T1]{fontenc}
  \usepackage[utf8]{inputenc}
  \usepackage{textcomp} % provide euro and other symbols
\else % if luatex or xetex
  \usepackage{unicode-math}
  \defaultfontfeatures{Scale=MatchLowercase}
  \defaultfontfeatures[\rmfamily]{Ligatures=TeX,Scale=1}
\fi
% Use upquote if available, for straight quotes in verbatim environments
\IfFileExists{upquote.sty}{\usepackage{upquote}}{}
\IfFileExists{microtype.sty}{% use microtype if available
  \usepackage[]{microtype}
  \UseMicrotypeSet[protrusion]{basicmath} % disable protrusion for tt fonts
}{}
\makeatletter
\@ifundefined{KOMAClassName}{% if non-KOMA class
  \IfFileExists{parskip.sty}{%
    \usepackage{parskip}
  }{% else
    \setlength{\parindent}{0pt}
    \setlength{\parskip}{6pt plus 2pt minus 1pt}}
}{% if KOMA class
  \KOMAoptions{parskip=half}}
\makeatother
\usepackage{xcolor}
\IfFileExists{xurl.sty}{\usepackage{xurl}}{} % add URL line breaks if available
\IfFileExists{bookmark.sty}{\usepackage{bookmark}}{\usepackage{hyperref}}
\hypersetup{
  pdftitle={253Project},
  pdfauthor={Jenny Li, Kristy Ma, Liz Cao},
  hidelinks,
  pdfcreator={LaTeX via pandoc}}
\urlstyle{same} % disable monospaced font for URLs
\usepackage[margin=1in]{geometry}
\usepackage{color}
\usepackage{fancyvrb}
\newcommand{\VerbBar}{|}
\newcommand{\VERB}{\Verb[commandchars=\\\{\}]}
\DefineVerbatimEnvironment{Highlighting}{Verbatim}{commandchars=\\\{\}}
% Add ',fontsize=\small' for more characters per line
\usepackage{framed}
\definecolor{shadecolor}{RGB}{248,248,248}
\newenvironment{Shaded}{\begin{snugshade}}{\end{snugshade}}
\newcommand{\AlertTok}[1]{\textcolor[rgb]{0.94,0.16,0.16}{#1}}
\newcommand{\AnnotationTok}[1]{\textcolor[rgb]{0.56,0.35,0.01}{\textbf{\textit{#1}}}}
\newcommand{\AttributeTok}[1]{\textcolor[rgb]{0.77,0.63,0.00}{#1}}
\newcommand{\BaseNTok}[1]{\textcolor[rgb]{0.00,0.00,0.81}{#1}}
\newcommand{\BuiltInTok}[1]{#1}
\newcommand{\CharTok}[1]{\textcolor[rgb]{0.31,0.60,0.02}{#1}}
\newcommand{\CommentTok}[1]{\textcolor[rgb]{0.56,0.35,0.01}{\textit{#1}}}
\newcommand{\CommentVarTok}[1]{\textcolor[rgb]{0.56,0.35,0.01}{\textbf{\textit{#1}}}}
\newcommand{\ConstantTok}[1]{\textcolor[rgb]{0.00,0.00,0.00}{#1}}
\newcommand{\ControlFlowTok}[1]{\textcolor[rgb]{0.13,0.29,0.53}{\textbf{#1}}}
\newcommand{\DataTypeTok}[1]{\textcolor[rgb]{0.13,0.29,0.53}{#1}}
\newcommand{\DecValTok}[1]{\textcolor[rgb]{0.00,0.00,0.81}{#1}}
\newcommand{\DocumentationTok}[1]{\textcolor[rgb]{0.56,0.35,0.01}{\textbf{\textit{#1}}}}
\newcommand{\ErrorTok}[1]{\textcolor[rgb]{0.64,0.00,0.00}{\textbf{#1}}}
\newcommand{\ExtensionTok}[1]{#1}
\newcommand{\FloatTok}[1]{\textcolor[rgb]{0.00,0.00,0.81}{#1}}
\newcommand{\FunctionTok}[1]{\textcolor[rgb]{0.00,0.00,0.00}{#1}}
\newcommand{\ImportTok}[1]{#1}
\newcommand{\InformationTok}[1]{\textcolor[rgb]{0.56,0.35,0.01}{\textbf{\textit{#1}}}}
\newcommand{\KeywordTok}[1]{\textcolor[rgb]{0.13,0.29,0.53}{\textbf{#1}}}
\newcommand{\NormalTok}[1]{#1}
\newcommand{\OperatorTok}[1]{\textcolor[rgb]{0.81,0.36,0.00}{\textbf{#1}}}
\newcommand{\OtherTok}[1]{\textcolor[rgb]{0.56,0.35,0.01}{#1}}
\newcommand{\PreprocessorTok}[1]{\textcolor[rgb]{0.56,0.35,0.01}{\textit{#1}}}
\newcommand{\RegionMarkerTok}[1]{#1}
\newcommand{\SpecialCharTok}[1]{\textcolor[rgb]{0.00,0.00,0.00}{#1}}
\newcommand{\SpecialStringTok}[1]{\textcolor[rgb]{0.31,0.60,0.02}{#1}}
\newcommand{\StringTok}[1]{\textcolor[rgb]{0.31,0.60,0.02}{#1}}
\newcommand{\VariableTok}[1]{\textcolor[rgb]{0.00,0.00,0.00}{#1}}
\newcommand{\VerbatimStringTok}[1]{\textcolor[rgb]{0.31,0.60,0.02}{#1}}
\newcommand{\WarningTok}[1]{\textcolor[rgb]{0.56,0.35,0.01}{\textbf{\textit{#1}}}}
\usepackage{graphicx}
\makeatletter
\def\maxwidth{\ifdim\Gin@nat@width>\linewidth\linewidth\else\Gin@nat@width\fi}
\def\maxheight{\ifdim\Gin@nat@height>\textheight\textheight\else\Gin@nat@height\fi}
\makeatother
% Scale images if necessary, so that they will not overflow the page
% margins by default, and it is still possible to overwrite the defaults
% using explicit options in \includegraphics[width, height, ...]{}
\setkeys{Gin}{width=\maxwidth,height=\maxheight,keepaspectratio}
% Set default figure placement to htbp
\makeatletter
\def\fps@figure{htbp}
\makeatother
\setlength{\emergencystretch}{3em} % prevent overfull lines
\providecommand{\tightlist}{%
  \setlength{\itemsep}{0pt}\setlength{\parskip}{0pt}}
\setcounter{secnumdepth}{-\maxdimen} % remove section numbering
\ifluatex
  \usepackage{selnolig}  % disable illegal ligatures
\fi

\title{253Project}
\author{Jenny Li, Kristy Ma, Liz Cao}
\date{2022/2/12}

\begin{document}
\maketitle

{
\setcounter{tocdepth}{2}
\tableofcontents
}
\hypertarget{part-a-b}{%
\section{Part a \& b}\label{part-a-b}}

\hypertarget{library-statements}{%
\subsection{Library statements}\label{library-statements}}

\begin{Shaded}
\begin{Highlighting}[]
\FunctionTok{library}\NormalTok{(dplyr)}
\FunctionTok{library}\NormalTok{(readr)}
\FunctionTok{library}\NormalTok{(broom)}
\FunctionTok{library}\NormalTok{(ggplot2)}
\FunctionTok{library}\NormalTok{(tidymodels) }
\FunctionTok{tidymodels\_prefer}\NormalTok{()}
\FunctionTok{theme\_set}\NormalTok{(}\FunctionTok{theme\_bw}\NormalTok{())       }
\FunctionTok{Sys.setlocale}\NormalTok{(}\StringTok{"LC\_TIME"}\NormalTok{, }\StringTok{"English"}\NormalTok{)}
\end{Highlighting}
\end{Shaded}

\begin{verbatim}
## [1] "English_United States.1252"
\end{verbatim}

\begin{Shaded}
\begin{Highlighting}[]
\FunctionTok{set.seed}\NormalTok{(}\DecValTok{74}\NormalTok{)}
\end{Highlighting}
\end{Shaded}

\hypertarget{read-in-data}{%
\subsection{Read in data}\label{read-in-data}}

\begin{Shaded}
\begin{Highlighting}[]
\NormalTok{breastCa}\OtherTok{\textless{}{-}}\FunctionTok{read\_csv}\NormalTok{(}\AttributeTok{file =} \StringTok{"breast{-}cancer.csv"}\NormalTok{)}
\end{Highlighting}
\end{Shaded}

\hypertarget{data-cleaning}{%
\subsection{Data cleaning}\label{data-cleaning}}

\begin{Shaded}
\begin{Highlighting}[]
\NormalTok{breastCa\_Re}\OtherTok{\textless{}{-}}\NormalTok{breastCa }\SpecialCharTok{\%\textgreater{}\%} 
  \FunctionTok{drop\_na}\NormalTok{() }\SpecialCharTok{\%\textgreater{}\%} 
  \FunctionTok{select}\NormalTok{(radius\_mean}\SpecialCharTok{:}\NormalTok{fractal\_dimension\_mean) }
\end{Highlighting}
\end{Shaded}

\hypertarget{creation-of-cv-folds}{%
\subsection{Creation of cv folds}\label{creation-of-cv-folds}}

\begin{Shaded}
\begin{Highlighting}[]
\NormalTok{breastCa\_Re\_CV}\OtherTok{\textless{}{-}}\FunctionTok{vfold\_cv}\NormalTok{(breastCa\_Re, }\AttributeTok{v =} \DecValTok{10}\NormalTok{)}
\end{Highlighting}
\end{Shaded}

\hypertarget{model-spec}{%
\subsection{Model spec}\label{model-spec}}

\begin{Shaded}
\begin{Highlighting}[]
\CommentTok{\#least square}
\NormalTok{lm\_spec }\OtherTok{\textless{}{-}}
    \FunctionTok{linear\_reg}\NormalTok{() }\SpecialCharTok{\%\textgreater{}\%} 
    \FunctionTok{set\_engine}\NormalTok{(}\AttributeTok{engine =} \StringTok{\textquotesingle{}lm\textquotesingle{}}\NormalTok{) }\SpecialCharTok{\%\textgreater{}\%} 
    \FunctionTok{set\_mode}\NormalTok{(}\StringTok{\textquotesingle{}regression\textquotesingle{}}\NormalTok{)}

\CommentTok{\#LASSO}
\NormalTok{lm\_lasso\_spec }\OtherTok{\textless{}{-}} 
  \FunctionTok{linear\_reg}\NormalTok{() }\SpecialCharTok{\%\textgreater{}\%}
  \FunctionTok{set\_args}\NormalTok{(}\AttributeTok{mixture =} \DecValTok{1}\NormalTok{, }\AttributeTok{penalty =} \FunctionTok{tune}\NormalTok{()) }\SpecialCharTok{\%\textgreater{}\%} \DocumentationTok{\#\# mixture = 1 indicates Lasso}
  \FunctionTok{set\_engine}\NormalTok{(}\AttributeTok{engine =} \StringTok{\textquotesingle{}glmnet\textquotesingle{}}\NormalTok{) }\SpecialCharTok{\%\textgreater{}\%} \CommentTok{\#note we are using a different engine}
  \FunctionTok{set\_mode}\NormalTok{(}\StringTok{\textquotesingle{}regression\textquotesingle{}}\NormalTok{)}
\end{Highlighting}
\end{Shaded}

\hypertarget{recipes-workflows}{%
\subsection{Recipes \& workflows}\label{recipes-workflows}}

\begin{Shaded}
\begin{Highlighting}[]
\CommentTok{\#least square}
\NormalTok{least\_rec }\OtherTok{\textless{}{-}} \FunctionTok{recipe}\NormalTok{(area\_mean }\SpecialCharTok{\textasciitilde{}}\NormalTok{ ., }\AttributeTok{data =}\NormalTok{ breastCa\_Re) }\SpecialCharTok{\%\textgreater{}\%}
    \FunctionTok{step\_corr}\NormalTok{(}\FunctionTok{all\_predictors}\NormalTok{()) }\SpecialCharTok{\%\textgreater{}\%} 
    \FunctionTok{step\_nzv}\NormalTok{(}\FunctionTok{all\_predictors}\NormalTok{()) }\SpecialCharTok{\%\textgreater{}\%} \CommentTok{\# removes variables with the same value}
    \FunctionTok{step\_normalize}\NormalTok{(}\FunctionTok{all\_numeric\_predictors}\NormalTok{()) }\SpecialCharTok{\%\textgreater{}\%} \CommentTok{\# important standardization step for LASSO}
    \FunctionTok{step\_dummy}\NormalTok{(}\FunctionTok{all\_nominal\_predictors}\NormalTok{())}

\NormalTok{least\_lm\_wf }\OtherTok{\textless{}{-}} \FunctionTok{workflow}\NormalTok{() }\SpecialCharTok{\%\textgreater{}\%}
    \FunctionTok{add\_recipe}\NormalTok{(least\_rec) }\SpecialCharTok{\%\textgreater{}\%}
    \FunctionTok{add\_model}\NormalTok{(lm\_spec)}
    
\CommentTok{\#LASSO}
\NormalTok{lasso\_wf}\OtherTok{\textless{}{-}} \FunctionTok{workflow}\NormalTok{() }\SpecialCharTok{\%\textgreater{}\%} 
  \FunctionTok{add\_recipe}\NormalTok{(least\_rec) }\SpecialCharTok{\%\textgreater{}\%}
  \FunctionTok{add\_model}\NormalTok{(lm\_lasso\_spec) }
\end{Highlighting}
\end{Shaded}

\hypertarget{fit-tune-models}{%
\subsection{Fit \& tune models}\label{fit-tune-models}}

\begin{Shaded}
\begin{Highlighting}[]
\CommentTok{\#least square}
\NormalTok{least\_fit }\OtherTok{\textless{}{-}} \FunctionTok{fit}\NormalTok{(least\_lm\_wf, }\AttributeTok{data =}\NormalTok{ breastCa\_Re) }

\NormalTok{least\_fit }\SpecialCharTok{\%\textgreater{}\%} \FunctionTok{tidy}\NormalTok{()}
\end{Highlighting}
\end{Shaded}

\begin{verbatim}
## # A tibble: 8 x 5
##   term                   estimate std.error statistic   p.value
##   <chr>                     <dbl>     <dbl>     <dbl>     <dbl>
## 1 (Intercept)             655.         2.10   311.    0        
## 2 radius_mean             369.         5.07    72.8   4.66e-288
## 3 texture_mean              1.63       2.29     0.713 4.76e-  1
## 4 smoothness_mean           0.746      3.11     0.239 8.11e-  1
## 5 compactness_mean        -69.7        6.51   -10.7   1.82e- 24
## 6 concavity_mean           37.5        5.54     6.76  3.56e- 11
## 7 symmetry_mean             1.65       2.79     0.594 5.53e-  1
## 8 fractal_dimension_mean   41.2        4.93     8.36  5.05e- 16
\end{verbatim}

\begin{Shaded}
\begin{Highlighting}[]
\CommentTok{\#LASSO}
\CommentTok{\#tune}
\NormalTok{penalty\_grid }\OtherTok{\textless{}{-}} \FunctionTok{grid\_regular}\NormalTok{(}
  \FunctionTok{penalty}\NormalTok{(}\AttributeTok{range =} \FunctionTok{c}\NormalTok{(}\SpecialCharTok{{-}}\DecValTok{3}\NormalTok{, }\DecValTok{1}\NormalTok{)), }\CommentTok{\#log10 transformed }
  \AttributeTok{levels =} \DecValTok{30}\NormalTok{)}

\NormalTok{tune\_output }\OtherTok{\textless{}{-}} \FunctionTok{tune\_grid}\NormalTok{( }\CommentTok{\# new function for tuning hyperparameters}
\NormalTok{  lasso\_wf, }\CommentTok{\# workflow}
  \AttributeTok{resamples =}\NormalTok{ breastCa\_Re\_CV, }\CommentTok{\# cv folds}
  \AttributeTok{metrics =} \FunctionTok{metric\_set}\NormalTok{(rmse, mae),}
  \AttributeTok{grid =}\NormalTok{ penalty\_grid }\CommentTok{\# penalty grid defined above}
\NormalTok{)}

\CommentTok{\#fit}
\NormalTok{best\_se\_penalty }\OtherTok{\textless{}{-}} \FunctionTok{select\_by\_one\_std\_err}\NormalTok{(tune\_output, }\AttributeTok{metric =} \StringTok{\textquotesingle{}mae\textquotesingle{}}\NormalTok{, }\FunctionTok{desc}\NormalTok{(penalty))}
\NormalTok{final\_wf\_se }\OtherTok{\textless{}{-}} \FunctionTok{finalize\_workflow}\NormalTok{(lasso\_wf, best\_se\_penalty)}
\NormalTok{lasso\_fit }\OtherTok{\textless{}{-}} \FunctionTok{fit}\NormalTok{(final\_wf\_se , }\AttributeTok{data =}\NormalTok{ breastCa\_Re)}
\NormalTok{lasso\_fit }\SpecialCharTok{\%\textgreater{}\%} \FunctionTok{tidy}\NormalTok{()}
\end{Highlighting}
\end{Shaded}

\begin{verbatim}
## # A tibble: 8 x 3
##   term                   estimate penalty
##   <chr>                     <dbl>   <dbl>
## 1 (Intercept)               655.     2.04
## 2 radius_mean               348.     2.04
## 3 texture_mean                0      2.04
## 4 smoothness_mean             0      2.04
## 5 compactness_mean          -26.3    2.04
## 6 concavity_mean             21.9    2.04
## 7 symmetry_mean               0      2.04
## 8 fractal_dimension_mean     14.4    2.04
\end{verbatim}

\hypertarget{part-c}{%
\section{Part c}\label{part-c}}

\hypertarget{calculate-and-collect-cv-metrics}{%
\subsection{Calculate and collect CV
metrics}\label{calculate-and-collect-cv-metrics}}

\begin{Shaded}
\begin{Highlighting}[]
\CommentTok{\# Least Square model}
\NormalTok{least\_fit\_cv }\OtherTok{\textless{}{-}} \FunctionTok{fit\_resamples}\NormalTok{(least\_lm\_wf,}
  \AttributeTok{resamples =}\NormalTok{ breastCa\_Re\_CV, }
  \AttributeTok{metrics =} \FunctionTok{metric\_set}\NormalTok{(rmse, mae)}
\NormalTok{)}

\NormalTok{least\_fit\_cv }\SpecialCharTok{\%\textgreater{}\%} \FunctionTok{collect\_metrics}\NormalTok{(}\AttributeTok{summarize =} \ConstantTok{TRUE}\NormalTok{)}
\end{Highlighting}
\end{Shaded}

\begin{verbatim}
## # A tibble: 2 x 6
##   .metric .estimator  mean     n std_err .config             
##   <chr>   <chr>      <dbl> <int>   <dbl> <chr>               
## 1 mae     standard    33.6    10    1.56 Preprocessor1_Model1
## 2 rmse    standard    49.4    10    4.90 Preprocessor1_Model1
\end{verbatim}

\begin{Shaded}
\begin{Highlighting}[]
\CommentTok{\# LASSO model}
\NormalTok{tune\_output }\SpecialCharTok{\%\textgreater{}\%} 
  \FunctionTok{collect\_metrics}\NormalTok{() }\SpecialCharTok{\%\textgreater{}\%} 
  \FunctionTok{filter}\NormalTok{(penalty }\SpecialCharTok{==}\NormalTok{ (best\_se\_penalty }
                     \SpecialCharTok{\%\textgreater{}\%} \FunctionTok{pull}\NormalTok{(penalty)))}
\end{Highlighting}
\end{Shaded}

\begin{verbatim}
## # A tibble: 2 x 7
##   penalty .metric .estimator  mean     n std_err .config              
##     <dbl> <chr>   <chr>      <dbl> <int>   <dbl> <chr>                
## 1    2.04 mae     standard    34.2    10    1.49 Preprocessor1_Model25
## 2    2.04 rmse    standard    51.1    10    5.26 Preprocessor1_Model25
\end{verbatim}

\hypertarget{part-d}{%
\section{Part d}\label{part-d}}

\hypertarget{residual-plots}{%
\subsection{Residual Plots}\label{residual-plots}}

\begin{Shaded}
\begin{Highlighting}[]
\NormalTok{least\_fit\_output }\OtherTok{\textless{}{-}}\NormalTok{ least\_fit }\SpecialCharTok{\%\textgreater{}\%}
  \FunctionTok{predict}\NormalTok{(}\AttributeTok{new\_data =}\NormalTok{ breastCa\_Re) }\SpecialCharTok{\%\textgreater{}\%}
  \FunctionTok{bind\_cols}\NormalTok{(breastCa\_Re) }\SpecialCharTok{\%\textgreater{}\%}
  \FunctionTok{mutate}\NormalTok{(}\AttributeTok{resid =}\NormalTok{ area\_mean }\SpecialCharTok{{-}}\NormalTok{ .pred)}

\FunctionTok{ggplot}\NormalTok{(least\_fit\_output, }\FunctionTok{aes}\NormalTok{(}\AttributeTok{x =}\NormalTok{ .pred, }\AttributeTok{y =}\NormalTok{ resid)) }\SpecialCharTok{+}
  \FunctionTok{geom\_point}\NormalTok{() }\SpecialCharTok{+}
  \FunctionTok{geom\_smooth}\NormalTok{() }\SpecialCharTok{+}
  \FunctionTok{geom\_hline}\NormalTok{(}\AttributeTok{yintercept =} \DecValTok{0}\NormalTok{, }\AttributeTok{color =} \StringTok{"red"}\NormalTok{) }\SpecialCharTok{+}
  \FunctionTok{labs}\NormalTok{(}\AttributeTok{x =} \StringTok{"Fitted values"}\NormalTok{, }\AttributeTok{y =} \StringTok{"Residuals"}\NormalTok{) }\SpecialCharTok{+}
  \FunctionTok{theme\_classic}\NormalTok{()}
\end{Highlighting}
\end{Shaded}

\includegraphics{STAT253-PROJECT_files/figure-latex/unnamed-chunk-10-1.pdf}

\begin{Shaded}
\begin{Highlighting}[]
\CommentTok{\# Residuals vs. predictors (x\textquotesingle{}s) }
\FunctionTok{ggplot}\NormalTok{(least\_fit\_output, }\FunctionTok{aes}\NormalTok{(}\AttributeTok{x =}\NormalTok{ concavity\_mean, }\AttributeTok{y =}\NormalTok{ resid)) }\SpecialCharTok{+}
  \FunctionTok{geom\_point}\NormalTok{() }\SpecialCharTok{+}
  \FunctionTok{geom\_smooth}\NormalTok{() }\SpecialCharTok{+}
  \FunctionTok{labs}\NormalTok{(}\AttributeTok{x =} \StringTok{"concavity mean"}\NormalTok{, }\AttributeTok{y =} \StringTok{"residual"}\NormalTok{) }\SpecialCharTok{+}
  \FunctionTok{geom\_hline}\NormalTok{(}\AttributeTok{yintercept =} \DecValTok{0}\NormalTok{, }\AttributeTok{color =} \StringTok{"red"}\NormalTok{) }\SpecialCharTok{+}
  \FunctionTok{theme\_classic}\NormalTok{()}
\end{Highlighting}
\end{Shaded}

\includegraphics{STAT253-PROJECT_files/figure-latex/unnamed-chunk-10-2.pdf}

\begin{Shaded}
\begin{Highlighting}[]
\CommentTok{\# Residuals vs. predictors (x\textquotesingle{}s) }
\FunctionTok{ggplot}\NormalTok{(least\_fit\_output, }\FunctionTok{aes}\NormalTok{(}\AttributeTok{x =}\NormalTok{ compactness\_mean, }\AttributeTok{y =}\NormalTok{ resid)) }\SpecialCharTok{+}
  \FunctionTok{geom\_point}\NormalTok{() }\SpecialCharTok{+}
  \FunctionTok{geom\_smooth}\NormalTok{() }\SpecialCharTok{+}
  \FunctionTok{labs}\NormalTok{(}\AttributeTok{x =} \StringTok{"compactness mean"}\NormalTok{, }\AttributeTok{y =} \StringTok{"residual"}\NormalTok{) }\SpecialCharTok{+}
  \FunctionTok{geom\_hline}\NormalTok{(}\AttributeTok{yintercept =} \DecValTok{0}\NormalTok{, }\AttributeTok{color =} \StringTok{"red"}\NormalTok{) }\SpecialCharTok{+}
  \FunctionTok{theme\_classic}\NormalTok{()}
\end{Highlighting}
\end{Shaded}

\includegraphics{STAT253-PROJECT_files/figure-latex/unnamed-chunk-10-3.pdf}

\begin{Shaded}
\begin{Highlighting}[]
\CommentTok{\# Residuals vs. predictors (x\textquotesingle{}s) }
\FunctionTok{ggplot}\NormalTok{(least\_fit\_output, }\FunctionTok{aes}\NormalTok{(}\AttributeTok{x =}\NormalTok{ fractal\_dimension\_mean, }\AttributeTok{y =}\NormalTok{ resid)) }\SpecialCharTok{+}
  \FunctionTok{geom\_point}\NormalTok{() }\SpecialCharTok{+}
  \FunctionTok{geom\_smooth}\NormalTok{() }\SpecialCharTok{+}
  \FunctionTok{labs}\NormalTok{(}\AttributeTok{x =} \StringTok{"fractal dimension mean"}\NormalTok{, }\AttributeTok{y =} \StringTok{"residual"}\NormalTok{) }\SpecialCharTok{+}
  \FunctionTok{geom\_hline}\NormalTok{(}\AttributeTok{yintercept =} \DecValTok{0}\NormalTok{, }\AttributeTok{color =} \StringTok{"red"}\NormalTok{) }\SpecialCharTok{+}
  \FunctionTok{theme\_classic}\NormalTok{()}
\end{Highlighting}
\end{Shaded}

\includegraphics{STAT253-PROJECT_files/figure-latex/unnamed-chunk-10-4.pdf}

\begin{Shaded}
\begin{Highlighting}[]
\CommentTok{\# Residuals vs. predictors (x\textquotesingle{}s) }
\FunctionTok{ggplot}\NormalTok{(least\_fit\_output, }\FunctionTok{aes}\NormalTok{(}\AttributeTok{x =}\NormalTok{ radius\_mean}\SpecialCharTok{*}\NormalTok{radius\_mean, }\AttributeTok{y =}\NormalTok{ resid)) }\SpecialCharTok{+}
  \FunctionTok{geom\_point}\NormalTok{() }\SpecialCharTok{+}
  \FunctionTok{geom\_smooth}\NormalTok{() }\SpecialCharTok{+}
  \FunctionTok{labs}\NormalTok{(}\AttributeTok{x =} \StringTok{"squared radius mean"}\NormalTok{, }\AttributeTok{y =} \StringTok{"residual"}\NormalTok{) }\SpecialCharTok{+}
  \FunctionTok{geom\_hline}\NormalTok{(}\AttributeTok{yintercept =} \DecValTok{0}\NormalTok{, }\AttributeTok{color =} \StringTok{"red"}\NormalTok{) }\SpecialCharTok{+}
  \FunctionTok{theme\_classic}\NormalTok{()}
\end{Highlighting}
\end{Shaded}

\includegraphics{STAT253-PROJECT_files/figure-latex/unnamed-chunk-10-5.pdf}

\begin{Shaded}
\begin{Highlighting}[]
\NormalTok{lasso\_fit\_output }\OtherTok{\textless{}{-}}\NormalTok{ lasso\_fit }\SpecialCharTok{\%\textgreater{}\%}
  \FunctionTok{predict}\NormalTok{(}\AttributeTok{new\_data =}\NormalTok{ breastCa\_Re) }\SpecialCharTok{\%\textgreater{}\%}
  \FunctionTok{bind\_cols}\NormalTok{(breastCa\_Re) }\SpecialCharTok{\%\textgreater{}\%}
  \FunctionTok{mutate}\NormalTok{(}\AttributeTok{resid =}\NormalTok{ area\_mean }\SpecialCharTok{{-}}\NormalTok{ .pred)}

\FunctionTok{ggplot}\NormalTok{(lasso\_fit\_output, }\FunctionTok{aes}\NormalTok{(}\AttributeTok{x =}\NormalTok{ .pred, }\AttributeTok{y =}\NormalTok{ resid)) }\SpecialCharTok{+}
  \FunctionTok{geom\_point}\NormalTok{() }\SpecialCharTok{+}
  \FunctionTok{geom\_smooth}\NormalTok{() }\SpecialCharTok{+}
  \FunctionTok{geom\_hline}\NormalTok{(}\AttributeTok{yintercept =} \DecValTok{0}\NormalTok{, }\AttributeTok{color =} \StringTok{"red"}\NormalTok{) }\SpecialCharTok{+}
  \FunctionTok{labs}\NormalTok{(}\AttributeTok{x =} \StringTok{"Fitted values"}\NormalTok{, }\AttributeTok{y =} \StringTok{"Residuals"}\NormalTok{) }\SpecialCharTok{+}
  \FunctionTok{theme\_classic}\NormalTok{()}
\end{Highlighting}
\end{Shaded}

\includegraphics{STAT253-PROJECT_files/figure-latex/unnamed-chunk-11-1.pdf}

\begin{Shaded}
\begin{Highlighting}[]
\CommentTok{\# Residuals vs. predictors (x\textquotesingle{}s) }
\FunctionTok{ggplot}\NormalTok{(lasso\_fit\_output, }\FunctionTok{aes}\NormalTok{(}\AttributeTok{x =}\NormalTok{ concavity\_mean, }\AttributeTok{y =}\NormalTok{ resid)) }\SpecialCharTok{+}
  \FunctionTok{geom\_point}\NormalTok{() }\SpecialCharTok{+}
  \FunctionTok{geom\_smooth}\NormalTok{() }\SpecialCharTok{+}
  \FunctionTok{labs}\NormalTok{(}\AttributeTok{x =} \StringTok{"concavity mean"}\NormalTok{, }\AttributeTok{y =} \StringTok{"residual"}\NormalTok{) }\SpecialCharTok{+}
  \FunctionTok{geom\_hline}\NormalTok{(}\AttributeTok{yintercept =} \DecValTok{0}\NormalTok{, }\AttributeTok{color =} \StringTok{"red"}\NormalTok{) }\SpecialCharTok{+}
  \FunctionTok{theme\_classic}\NormalTok{()}
\end{Highlighting}
\end{Shaded}

\includegraphics{STAT253-PROJECT_files/figure-latex/unnamed-chunk-11-2.pdf}

\begin{Shaded}
\begin{Highlighting}[]
\CommentTok{\# Residuals vs. predictors (x\textquotesingle{}s) }
\FunctionTok{ggplot}\NormalTok{(lasso\_fit\_output, }\FunctionTok{aes}\NormalTok{(}\AttributeTok{x =}\NormalTok{ compactness\_mean, }\AttributeTok{y =}\NormalTok{ resid)) }\SpecialCharTok{+}
  \FunctionTok{geom\_point}\NormalTok{() }\SpecialCharTok{+}
  \FunctionTok{geom\_smooth}\NormalTok{() }\SpecialCharTok{+}
  \FunctionTok{labs}\NormalTok{(}\AttributeTok{x =} \StringTok{"compactness mean"}\NormalTok{, }\AttributeTok{y =} \StringTok{"residual"}\NormalTok{) }\SpecialCharTok{+}
  \FunctionTok{geom\_hline}\NormalTok{(}\AttributeTok{yintercept =} \DecValTok{0}\NormalTok{, }\AttributeTok{color =} \StringTok{"red"}\NormalTok{) }\SpecialCharTok{+}
  \FunctionTok{theme\_classic}\NormalTok{()}
\end{Highlighting}
\end{Shaded}

\includegraphics{STAT253-PROJECT_files/figure-latex/unnamed-chunk-11-3.pdf}

\begin{Shaded}
\begin{Highlighting}[]
\CommentTok{\# Residuals vs. predictors (x\textquotesingle{}s) }
\FunctionTok{ggplot}\NormalTok{(lasso\_fit\_output, }\FunctionTok{aes}\NormalTok{(}\AttributeTok{x =}\NormalTok{ fractal\_dimension\_mean, }\AttributeTok{y =}\NormalTok{ resid)) }\SpecialCharTok{+}
  \FunctionTok{geom\_point}\NormalTok{() }\SpecialCharTok{+}
  \FunctionTok{geom\_smooth}\NormalTok{() }\SpecialCharTok{+}
  \FunctionTok{labs}\NormalTok{(}\AttributeTok{x =} \StringTok{"fractal dimension mean"}\NormalTok{, }\AttributeTok{y =} \StringTok{"residual"}\NormalTok{) }\SpecialCharTok{+}
  \FunctionTok{geom\_hline}\NormalTok{(}\AttributeTok{yintercept =} \DecValTok{0}\NormalTok{, }\AttributeTok{color =} \StringTok{"red"}\NormalTok{) }\SpecialCharTok{+}
  \FunctionTok{theme\_classic}\NormalTok{()}
\end{Highlighting}
\end{Shaded}

\includegraphics{STAT253-PROJECT_files/figure-latex/unnamed-chunk-11-4.pdf}

\begin{Shaded}
\begin{Highlighting}[]
\CommentTok{\# Residuals vs. predictors (x\textquotesingle{}s) }
\FunctionTok{ggplot}\NormalTok{(lasso\_fit\_output, }\FunctionTok{aes}\NormalTok{(}\AttributeTok{x =}\NormalTok{ radius\_mean}\SpecialCharTok{*}\NormalTok{radius\_mean, }\AttributeTok{y =}\NormalTok{ resid)) }\SpecialCharTok{+}
  \FunctionTok{geom\_point}\NormalTok{() }\SpecialCharTok{+}
  \FunctionTok{geom\_smooth}\NormalTok{() }\SpecialCharTok{+}
  \FunctionTok{labs}\NormalTok{(}\AttributeTok{x =} \StringTok{"squared radius mean"}\NormalTok{, }\AttributeTok{y =} \StringTok{"residual"}\NormalTok{) }\SpecialCharTok{+}
  \FunctionTok{geom\_hline}\NormalTok{(}\AttributeTok{yintercept =} \DecValTok{0}\NormalTok{, }\AttributeTok{color =} \StringTok{"red"}\NormalTok{) }\SpecialCharTok{+}
  \FunctionTok{theme\_classic}\NormalTok{()}
\end{Highlighting}
\end{Shaded}

\includegraphics{STAT253-PROJECT_files/figure-latex/unnamed-chunk-11-5.pdf}
\# Part e \textgreater{} In the OLS, ``radius\_mean'' is the most
important predictor for our quantitative outcome since it has the lowest
p.value. This shows the result is statistically significant and we have
stronger evidence to reject the null hypothesis and instead claim a
possible association between the mean area of tumor and the mean radius
of the tumor.

\begin{quote}
In the LASSO, ``radius\_mean'', ``compactness\_mean'',
``concavity\_mean'', and ``fractal\_dimension\_mean'', are the most
important predictors of our quantitative outcome. The penalty shrinks
the coefficients of three variables to zero due to their minimum effect
on the response variable, and keeps important predictors in the model.
\end{quote}

\begin{quote}
The methods we've applied did not reach the consensus on the most
important variable. The strong association between the mean area of
tumor and the mean radius of the tumor, indicated by the lowest p-value
for ``radius\_mean'' and its inclusion in the LASSO model, is expected
by their definition.
\end{quote}

\hypertarget{summarize-investigations-decide-on-an-overall-best-model-based-on-your-investigations-so-far.-to-do-this-make-clear-your-analysis-goals.-predictive-accuracy-interpretability-a-combination-of-both}{%
\subsection{2. Summarize investigations: Decide on an overall best model
based on your investigations so far. To do this, make clear your
analysis goals. Predictive accuracy? Interpretability? A combination of
both?}\label{summarize-investigations-decide-on-an-overall-best-model-based-on-your-investigations-so-far.-to-do-this-make-clear-your-analysis-goals.-predictive-accuracy-interpretability-a-combination-of-both}}

\begin{quote}
Answer: Our goal is to have both accurate predictions and keep
interpretability. Our proposed best model: area\_mean\textasciitilde{}
radius\_mean + compactness\_mean + concavity\_mean +
fractal\_dimension\_mean.( based on the p\_value and the coefficents
from LASSO. We prefer to keep all the variables that contain cofficients
because the total of 4 variabls for a model is interpretable.) However,
we are using the mean of radius to predict the area which does not
really make sense in implementation, but for this assignment we will
just leave it there. Also, this is coherent with the residual plot of
radius which is a non-linear relationship.
\end{quote}

\hypertarget{societal-impact-are-there-any-harms-that-may-come-from-your-analyses-andor-how-the-data-were-collected-what-cautions-do-you-want-to-keep-in-mind-when-communicating-your-work}{%
\subsection{3. Societal impact: Are there any harms that may come from
your analyses and/or how the data were collected? What cautions do you
want to keep in mind when communicating your
work?}\label{societal-impact-are-there-any-harms-that-may-come-from-your-analyses-andor-how-the-data-were-collected-what-cautions-do-you-want-to-keep-in-mind-when-communicating-your-work}}

\begin{quote}
Answer: There are some harms that may occur, such as some patients may
not be willing to provide their personal information or records for the
case study. We need to protect the information safety of the patient.
Also, we should know the number of observations may not be able to be
generalized.
\end{quote}

\end{document}
